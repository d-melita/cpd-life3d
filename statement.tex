%% -*- Mode:LaTeX -*-
%% -*- Module: projPDC-life3d.tex

%% Enunciado do Projecto de CPD 2023/2024, P3

%%%%%%%%%%%%%%%%%%%%%%%%%%%%%%%%%%%%%%%%%%%%%%%%%%%%%%%%%%%%%%%%%%%%%%%%%%%%%%
%%

\documentclass[11pt,a4paper,twoside]{article}

%%\usepackage[portuges]{babel}
%%\usepackage[latin1]{inputenc}%
\usepackage{anysize}            % Para ter um area de texto maior %
\usepackage{times}
% \usepackage{html}
\usepackage{amsmath}
\usepackage{bm}
\usepackage{a4wide}
\usepackage{soul}
\usepackage{caption}
\usepackage{subcaption}
\usepackage{enumitem}
\usepackage{algorithm}
\usepackage{algpseudocode}
\usepackage{graphicx}
\usepackage{tikz}
\usetikzlibrary{shapes,arrows}

\usepackage{fancyhdr}
\pagestyle{fancy}
\fancyhead[LE,RO]{\slshape \leftmark}
\fancyhead[LO,RE]{}
\fancyfoot[C]{\thepage}

\usepackage{tikz}
\usepackage{tikz-3dplot}
\usetikzlibrary{3d}
\pgfmathsetmacro{\radius}{0.0}
\newcommand{\frontcolor}{white}
\newcommand{\rightcolor}{white}
\newcommand{\leftcolor}{white}

\renewcommand{\footrulewidth}{0.4pt}

\newcommand{\ie}{{\it i.e.}}
\newcommand{\eg}{{\it e.g.}}

\pdfinfo{ % Info dictionary of PDF output; all keys are optional.
    %% /Author (Jos� Monteiro)
    %% /CreationDate (D:201409271030)      % (D:YYYYMMDDhhmmss)
                                           % default: the actual date
    %% /ModDate (D:201109261530)           % ModDate is similar
    %% /Creator (TeX)                      % default: "TeX"
    %% /Producer (pdfTeX)                  % default: "pdfTeX"
    /Title (Game of Life 3D)            %
    /Subject (PDC Project Assignment - 2023/2024) %
    /Keywords (computacao paralela distribuida parallel distributed computing ) %
}

\makeatletter
\long\def\@makecaption#1#2{
   \vskip 10pt
   \setbox\@tempboxa\hbox{\textbf{#1: #2}}
   \ifdim \wd\@tempboxa >\hsize % IF longer than one line:
       \textbf{#1: #2}\par                %   THEN set as ordinary paragraph.
     \else                      %   ELSE  center.
       \hbox to\hsize{\hfil\box\@tempboxa\hfil}
   \fi}
\makeatother

\def\floatpagefraction{1}
\def\textfraction{0}
\def\topfraction{1}
\def\bottomfraction{1}


%%%%%%%%%%%%%%%%%%%%%%%%%%%%%%%%%%%%%%%%%%%%%%%%%%%%%%%%%%%%%%%%%%%%%%%%%%%%%%

\begin{document}


\title{
  \vspace*{-4cm}
  \begin{minipage}[l]{0cm}
    \hspace*{-4.5cm}
    %%\vspace*{-2cm}
    \includegraphics[width=6cm]{IST_A_CMYK_Pos.pdf}
  \end{minipage}
  \begin{minipage}[r]{10cm}
    \begin{center}
    \vspace*{2.5cm}
%      \LARGE Instituto Superior T\'{e}cnico\\[0.5cm]
%%      \hspace*{-1.5cm}\Large MEIC - Mestrado em Engenharia
%%      Inform�tica e de Computadores\\[0.5cm]
%%      \hspace*{-1.5cm}\Large MEEC - Mestrado em Engenharia
%%      Electrot�cnica e de Computadores\\[0.5cm]
%%      \hspace*{-1.5cm}\Large MERC - Mestrado em Engenharia
%%      de Redes de Comunica��o\\[2.0cm]
    \end{center}
  \end{minipage}
}
\author{
  \vspace*{1.5cm}
  \Huge Parallel and Distributed Computing\\[1.5cm]
  \LARGE \underline{Project Assignment}\\[1.5cm]
%%  \includegraphics[width=12cm]{figs/network-flow-graph-02.png}\\[1.0cm]
%%  \includegraphics[width=12cm]{figs/network-flow-graph.pdf}\\[1.0cm]
  \vspace*{8cm}
  \fbox{\parbox{13cm}{
      \begin{center}
        \huge
        \underline{{\sc Game of Life 3D}}
      \end{center}
    }}\\[0.5cm]
  \Large \textbf{Version 1.0 (06/02/2024)} } \date{\vspace*{1cm}
  \Large 2023/2024\\[0.25cm]
  3rd Quarter}

\maketitle
\thispagestyle{empty}

\newpage
\tableofcontents
\setcounter{page}{1}

\vspace*{4cm}
\flushbottom
\noindent \textbf{\Large Revisions}\\[1cm]
\begin{tabular}{|ll|}
  \hline
  & \\
  Version 1.0 (February 6, 2024) & Initial Version \\
  \hline
\end{tabular}

\newpage

%%%%%%%%%%%%%%%%%%%%%%%%%%%%%%%%%%%%%%%%%%%%%%%%%%%%%%%%%%%%%%%%%%%%%%%%%%%%%%

\section{Introduction}
\label{sec:introduction}

The purpose of this class project is to gain experience in parallel
programming on shared and distributed memory systems, using OpenMP and
MPI, respectively.  For this assignment you are to write a sequential
and two parallel implementations of the game of life in 3D.


%%%%%%%%%%%%%%%%%%%%%%%%%%%%%%%%%%%%%%%%%%%%%%%%%%%%%%%%%%%%%%%%%%%%%%%%%%%%%%

\section{Problem Description}
\label{sec:problem-description}

The life game is a zero-player game, that starts from an initial
distribution of individuals in the game space. The game evolves by
computing new generations of individuals based on simple rules.

For this assignment we consider a three-dimensional space, a cube with
$N$ cells per side. Each cell will have 26 neighbor cells (this
is true for all cells, we consider that the sides wrap-around, \ie,
cells with indices $i$ and $i+N$ are the same) . Each cell of this
cube may be occupied (live) or empty (dead). In order to determine the
status of a cell in the next generation, the following rules should be used:
\begin{itemize}
\item a live cell with 4 or less live neighbors dies.
\item a live cell with 5 to 13 live neighbors lives on to the next generation.
\item a live cell with more than 13 live neighbors dies.
\item a dead cell with 7 to 10 live neighbors becomes a live cell.
\end{itemize}

Additionally, consider that we may have up to 9 species in the
game, thus the cells will hold a value between 1 and 9. The species of
a cell that becomes alive corresponds to the majority of the species
surrounding it (to break ties, use the lower species value in the tie).

From an initial population in the cube, the objective is to compute
the population through a given number of generations. Note that all
the cells of the next generation are computed from the cells of the
previous iterations, hence the order that the cells are computed is irrelevant.

To avoid reading (and moving around) large files, the initial
population is generated randomly inside the program. To ensure that
everyone has the same starting point (allowing for the validation of
the results), the routine to create the simulation space is given in
Appendix~\ref{sec:init_code}. You are free to modify this this code, but
please make sure you get the same initial population!


%%%%%%%%%%%%%%%%%%%%%%%%%%%%%%%%%%%%%%%%%%%%%%%%%%%%%%%%%%%%%%%%%%%%%%%%%%%%%%

\section{Implementation Details}

\subsection{Input Data}
\label{sec:input-data}

Your program should allow exactly 4 input line parameters, in this order:
\begin{itemize}
\item number of generations (positive integer)
\item number of cells per side of the cube (positive integer)
\item density of the initial population (float between 0 and 1)
\item seed for the random number generator (integer)
\end{itemize}


\subsection{Output Data}
\label{sec:output-data}

The program should send to the standard output ({\tt stdout}) 9 lines,
one per species and in increasing order of these, each with three
integers representing, respectively: the species id; the maximum
number of these species over all generations; generation where this
maximum was achieved (lowest generation in case of draw).

Note that we will be using very large simulation spaces, be sure to
use {\bf integers with 64 bits} for your counters.

Your program should send these output lines (and {\bf nothing else!}) to the
standard output.

The project \textbf{cannot be graded} unless you follow these input
and output rules!


\subsection{Sample Problem}
\label{sec:sample-problem}

If you run your program with the following input parameters, the
result should be what is presented next:
\begin{verbatim}
$ ./life3d 4 4 .4 100
1 4 3
2 1 0
3 2 0
4 1 0
5 1 0
6 14 4
7 5 0
8 2 0
9 4 0
\end{verbatim}

For debugging purposes, you can check the population evolution for this
example in Appendix~\ref{sec:debug}.


\subsection{Measuring Execution Time}
\label{sec:exec-time}

To make sure everyone uses the same measure for the execution time,
the routine {\tt omp\_get\_wtime()} from OpenMP will be used, which
measures real time (also known as ``wall-clock time''). This same
routine should be used by all three versions of your project.

Hence, your programs should have a structure similar to this:

\begin{small}
\begin{verbatim}
#include <omp.h>
<...>

int main(int argc, char *argv[])
{
    double exec_time;

    grid = gen_initial_grid(N, seed, density);
    exec_time = -omp_get_wtime();

    simulation();

    exec_time += omp_get_wtime();
    fprintf(stderr, "%.1fs\n", exec_time);

    print_result();       // to the stdout!   
 }
\end{verbatim}
\end{small}

In this way the execution time will only account the algorithm running
time, and be sent to the standard error, {\tt stderr}. The use of
these two output streams allows the validation of the results and
analysis of execution time needed to be performed separately.

Because this time routine is part of OpenMP, you need the include {\tt
  omp.h} and compile all your programs with the flag {\tt -fopenmp}.


%%%%%%%%%%%%%%%%%%%%%%%%%%%%%%%%%%%%%%%%%%%%%%%%%%%%%%%%%%%%%%%%%%%%%%%%%%%%%%
%%

\section{Submission}

\subsection{Part 1 - Serial implementation}
\label{sec:part1-serial}

Write a serial implementation of the algorithm in C (or C++). Name the
source file of this implementation {\tt life3d.c}. As stated above, your
program should expect exactly four input parameters.

This will be your base for comparisons and it is expected that it
is as efficient as possible.


%%%%%%%%%%%%%%%%%%%%%%%%%%%%%%%%%%%%%%%%%%%%%%%%%%%%%%%%%%%%%%%%%%%%%%%%%%%%%%
%%

\subsection{Part 2 - OpenMP implementation}
\label{sec:part1-openmp}

Write an OpenMP implementation of the algorithm, with the same rules
and input/output descriptions. Name this source code {\tt
  life3d-omp.c}. You can start by simply adding OpenMP directives, but
you are free, and encouraged, to modify the code in order to make the
parallelization more effective and more scalable.  Be careful about
synchronization and load balancing!


%%%%%%%%%%%%%%%%%%%%%%%%%%%%%%%%%%%%%%%%%%%%%%%%%%%%%%%%%%%%%%%%%%%%%%%%%%%%%%
%%

\subsection{Part 3 - MPI implementation}
\label{sec:part1-mpi}

Write an MPI implementation of the algorithm as for OpenMP, and
address the same issues. Name this source code {\tt life3d-mpi.c}.

For MPI, you will need to modify your code substantially. Besides
synchronization and load balancing, you will need to take into
account the minimization of the impact of communication costs. You are
encouraged to explore different approaches for the problem decomposition.

Note that this distributed version should permit running larger
instances, namely instances that do not fit in the memory of a single
machine.

Extra credits will be given to groups that present a combined
MPI+OpenMP implementation.

%%%%%%%%%%%%%%%%%%%%%%%%%%%%%%%%%%%%%%%%%%%%%%%%%%%%%%%%%%%%%%%%%%%%%%%%%%%%%%%
%%%

\subsection{What to Turn In}
\label{sec:turn-in}

You must eventually submit the sequential and both parallel versions
of your program ({\bf please use the filenames indicated above}), and
a table with the times to run the parallel versions on input data that
will be made available (for 1, 2, 4 and 8 parallel tasks for both
OpenMP and MPI, and additionally 16, 32 and 64 for MPI).

For both the OpenMP and MPI versions (not for serial), you must also
submit a short report about the results (2-4 pages) that 
discusses:\\[-6mm]
%%
\begin{itemize}
  %%
  \parskip 0pt
  %%
\item the approach used for parallelization
  %%
%% \item what algorithm was used to address the problem
%%   %%
\item what decomposition was used
  %%
\item what were the synchronization concerns and why
  %%
\item how was load balancing addressed
  %%
\item what are the performance results, and are they what you expected
  %%
\end{itemize}

The code, makefile and report will be uploaded to the Fenix system in
a zip file. {\bf Name these files} as {\tt g<n>serial.zip}, {\tt
  g<n>omp.zip} and {\tt g<n>mpi.zip}, where {\tt <n>} is your group number.
%%

\subsection{Deadlines}
\begin{description}
  %%
  \parskip 0pt
  %%
\item[] 1st due date, serial version: {\bf March 2nd}, until 23:59.\\
%  Note: your project will be tested in the lab class right after this due date.
  %%
\item[] 2nd due date (OpenMP):  {\bf March 16th}, until 23:59.\\
%  Note: your project will be tested in the lab class right after this due date.
  %%
\item[] 3rd due date (MPI):  {\bf March 20th}, until 23:59.\\
%  Note: your project will be tested in the lab class right after this due date.
  %%
\end{description}



%%%%%%%%%%%%%%%%%%%%%%%%%%%%%%%%%%%%%%%%%%%%%%%%%%%%%%%%%%%%%%%%%%%%%%%%%%%%%%%

\newpage
\appendix

\section{Code to Generate the Initial Grid}
\label{sec:init_code}

\begin{verbatim}
#define N_SPECIES 9

unsigned int seed;

void init_r4uni(int input_seed)
{
    seed = input_seed + 987654321;
}

float r4_uni()
{
    int seed_in = seed;

    seed ^= (seed << 13);
    seed ^= (seed >> 17);
    seed ^= (seed << 5);

    return 0.5 + 0.2328306e-09 * (seed_in + (int) seed);
}

char ***gen_initial_grid(long long N, float density, int input_seed)
{
    int x, y, z;

    grid = (char ***) malloc(N * sizeof(char **));
    if(grid == NULL) {
        printf("Failed to allocate matrix\n");
        exit(1);
    }
    for(x = 0; x < N; x++) {
        grid[x] = (char **) malloc(N * sizeof(char *));
        if(grid[x] == NULL) {
            printf("Failed to allocate matrix\n");
            exit(1);
        }
        grid[x][0] = (char *) calloc(N * N, sizeof(char));
        if(grid[x][0] == NULL) {
            printf("Failed to allocate matrix\n");
            exit(1);
        }
        for (y = 1; y < N; y++)
            grid[x][y] = grid[x][0] + y * N;
    }

    init_r4uni(input_seed);
    for (x = 0; x < N; x++)
        for (y = 0; y < N; y++)
            for (z = 0; z < N; z++)
                if(r4_uni() < density)
                    grid[x][y][z] = (int)(r4_uni() * N_SPECIES) + 1;

    return grid;
}
\end{verbatim}


%%%%%%%%%%%%%%%%%%%%%%%%%%%%%%%%%%%%%%%%%%%%%%%%%%%%%%%%%%%%%%%%%%%%%%%%%%%%%%%

\newpage
\section{Debug of Simple Input}
\label{sec:debug}

Your program \textbf{should not} print this, use this information only
for debug purposes.

\begin{verbatim}
Generation 0   ------------------------------
Layer 0:
    1 6 
  1 8   
  6   4 
    9   

Layer 1:
6       
1   2   
3   9 7 
6       

Layer 2:
  3 9 6 
        
9   6   
6       

Layer 3:
  6   6 
6 7 7   
  8 7 5 
7       

Generation 1   ------------------------------
Layer 0:
6   1 6 
        
      4 
    9   

Layer 1:
6   1   
1   2   
3   9 7 
6       

Layer 2:
  3 9 6 
        
9   6   
6       

Layer 3:
6 6   6 
6 7     
6 8 7 5 
7       

Generation 2   ------------------------------
Layer 0:
6   1 6 
        
    7 4 
    9   

Layer 1:
6   1   
1   2   
3 6 9 7 
6       

Layer 2:
  3 9 6 
        
9   6   
        

Layer 3:
6 6 6 6 
6 7     
6 8 7 5 
7       

Generation 3   ------------------------------
Layer 0:
6   1 6 
        
    7 4 
        

Layer 1:
6 1 1   
1   2   
3 6 9 7 
6       

Layer 2:
  3 9 6 
        
9   6   
        

Layer 3:
6 6 6 6 
6       
6 8 7 5 
7       

Generation 4   ------------------------------
Layer 0:
6   1 6 
        
    7 4 
        

Layer 1:
6 1 1 6 
1   2   
3 6 9 7 
6       

Layer 2:
  3 9 6 
        
9   6   
        

Layer 3:
6 6 6 6 
6       
6 8 7 5 
7
\end{verbatim}


\end{document}


%%% Local Variables:
%%% mode: latex
%%% TeX-master: t
%%% eval: (auto-fill-mode t)
%%% End:

