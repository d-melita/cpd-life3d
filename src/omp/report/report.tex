\documentclass{article}

\usepackage{amsmath, amsthm, amssymb, amsfonts}
\usepackage{thmtools}
\usepackage{graphicx}
\usepackage{setspace}
\usepackage{geometry}
\usepackage{float}
\usepackage{hyperref}
\usepackage[utf8]{inputenc}
\usepackage[english]{babel}
\usepackage{framed}
\usepackage[dvipsnames]{xcolor}
\usepackage{tcolorbox}

\newcommand{\HRule}[1]{\rule{\linewidth}{#1}}

% ------------------------------------------------------------------------------

\begin{document}

% ------------------------------------------------------------------------------
% Cover Page and ToC
% ------------------------------------------------------------------------------

\title{ \normalsize \textsc{}
		\\ [2.0cm]
		\HRule{1.5pt} \\
		\LARGE \textbf{Parallel and Distributed Computing
		\HRule{2.0pt} \\ [0.6cm] \LARGE{Project Report - OpenMP Delivery} \vspace*{10\baselineskip}}
		}
\date{}
\author{\textbf{Authors} \\ 
		Carolina Coelho - 99189\\
		Diogo Melita - 99202\\
		Diogo Antunes - 99210}

\maketitle
\newpage

% ------------------------------------------------------------------------------

\section{Introduciton}

For this stage of the project, we are asked to optimize our 
serial version using OpenMP. We will be using the same 
implementation as the previous stage, but we will be adding 
OpenMP directives to parallelize the code. 

\section{Approach Used}

The first step was to identify the parts of the code that 
could be parallelized. By analyzing the code, we concluded that 
the most time-consuming part was on the function \texttt{simulation}. 
This function is responsible for simulating the multiple generations, 
and therefore, it is very time consuming.

Upon identifying the part of the code that could be parallelized, 
we identified the 2 nested loops that needed to be took care of in 
order to speed up the code: the nested loop that calculates the initial statistics 
of the grid and the loop that calculates the next generations of the grid. 

Additionally, we also used \textbf{VTune} to profile the code and 
identify the most time-consuming parts of the code in order to ease 
the parallelization process.

\section{OpenMP Parallelization}
To parallelize the code, we used the OpenMP directives, more specifically  
the following directives: \texttt{parallel}, \texttt{for}, \texttt{collapse}, 
\texttt{reduction}, and \texttt{single}.

The \texttt{parallel} directive is used to create a team of threads 
that will execute the code inside the block. In this block, we have 
all the code mentioned above that we wanted to parallelize. 

To parallelize the loops, we combined the \texttt{for}, \texttt{collapse}, 
and \texttt{reduction} directives. The \texttt{for} directive is self-explanatory, 
the \texttt{collapse} directive is used to collapse nested loops into one, in 
this case we used it to collapse the 3 loops since the grid is 3 dimentional.
Finally, the \texttt{reduction} directive is used to perform a reduction 
operation on the variables that are shared between the threads. In this case, we 
used this directive with the \texttt{+} operator to sum the number of alive cells 
of each specie in the grid. Since the parallelization we intended to do was having 
multiple threads not only calulating the next generation of the grid, but also the 
number of alive cells of each specie we used the \texttt{reduction} directive as it 
makes a private copy for each thread of the variable specified and then combines 
the results at the end, thus achieving the intended result. 

Finally, we used the \texttt{single} directive to make sure that only one thread 
executed parts of the code inside the block that shouldn't be parallelized. This 
parts were responsible for swapping the pointers of the old and new grids and calculating the 
stats of each generation and therefore, it was important to make sure that only 
one thread executed this part of the code to achieve the correct results.

\section{Results}

To test the parallelized code, we used available input files the course's page 
and the lab's machines. We were also careful to run this OpenMP version with the 
environment variable \texttt{OMP\_NUM\_THREADS} set to number of physical cores 
of the machine we were running the code on. 

We compared the results of the parallelized code with the serial version and 
concluded that the parallelized version was much faster than the serial version. 
The speedup achieved was significant and very close to the ideal speedup, the number of 
physical cores.

\newpage

% ------------------------------------------------------------------------------
% Reference and Cited Works
% ------------------------------------------------------------------------------

\bibliographystyle{IEEEtran}
% \bibliography{References.bib}

% ------------------------------------------------------------------------------

\end{document}
